\title{Financial Macro/Micro Foundations}

This work is a collaboration between two Sloan Foundation funded teams
[and another team. May they introduce themselves and their agenda here…]

One team is building tools to explore market microstructure and the
micro-foundations of information processing in the financial sector,
especially the attention and behavior of market actors.

\section{Components}

\subsection{The market model}

There are two potential ABM models that we could use as starting points:
1) the emini model (based on the NetLogo ABM model developed by Mark Paddrik
while he was working at the CFTC and which was subsequently calibrated and
then used to mimic the S&P 500 emini futures market around the time of the
“Flash Crash”, May 2010) and 2) the HFT model (built by Zak and collaborators
from the HFT world, used at one point as a back-testing framework for
implementing the potential profitability of trading strategies and which has
a high degree of flexibility and validity with respect to the current
real-world Market Micro-structure).

The emini model has the advantage that it is in the public domain (which
satisfies one of the criteria of Sloan Foundation grants), it is also
relatively simple (with only three different types of agents) and,
using the package PyNetLogo, can be wrapped in such a way that it
should be able to readily interact with HARK software. 

It has the disadvantage that it is not clear how a consumer (from the
HARK world) who wishes to change their allocation of wealth to the risky
asset would translate that desire into orders that would then appear
in the order book. This mapping from a desired delta in wealth, at
a set price which would result in a set number of shares to buy or sell,
to a change in the agent parameters in the emini ABM market world is
an open question. One simple way that we could implement this mapping
is to have a change in demand from an agent translate into a change
in the percentage of buy orders vs sell orders randomly sent to the
market by the Liquidity Demanders (who are the closest agent to a
Fundamental Buyer/Seller in the emini ABM). 

Another way that the HARK model could interact with the emini ABM would
be with regards to the funding available to the two market facilitators
(the Liquidity Suppliers and the Market Makers). For example, if the
interest rate increases, then the cost of capital could result in
fewer Market Makers or vice versa if the interest rate decreases
(reflecting an decreases or increases in liquidity).

The HFT model has the advantage that in addition to a variety of
market makers type agents and opportunisti type agents, it has brokers
who use HFT algorithms that will translate a customer’s ‘macro’ orders
into actual positions in the risky asset order book, in a way that is
‘optimal’ for the customer. This provides a straightforward mapping
from a customer’s desired delta in how much of their wealth is in
the risky asset to positions in the order book.

In a similar manner to the emini market world, changes in funding can
result in changes in capital to market makers and other liquidity
providers. (This idea is probably secondary to the problem of getting
consumer’s ‘macro’ orders translated into orderbook positions.)

SUMMARY: A consumer decides to change the amount of wealth that they
hold in the risky asset--this is the ‘macro’ order. In the emini model
this is translated into the percentage of liquidity demanders who want
to place bid vs ask orders. In the HFT model these ‘macro’ orders are
sent to a broker who then uses ‘optimal’ HFT algorithms to place
orders in the order book. In both market models there are additional
agents (liquidity providers and potentially opportunist traders)
who regularly place orders in the order book, allowing the market to function.

\subsection{The macro model}

\subsubsection{Portfolio construction}

In the macroeconomic model implemented using HARK:
N consumers earn, save, get returns on savings, consume, and die
over a discrete time period
Earnings are subject to exogenous shocks
Each consumer has an optimal consumer rate that depends on their state, which
includes their wealth, level of permanent income, age, and so on.
Each consumer also has an optimal portfolio allocation for their savings,
which may be stored in assets of various expected return and risk.

The consumer’s decision problem is modeled as a Markov Decision Process and
by applying dynamic programming to the Bellman form of the model.

It’s useful to note that the Macro model here is integrating aspects of
portfolio theory. In principle, the model can be extended to a range of
assets, each with different expected returns and variance. At this time,
HARK does not yet handle the possibility of correlated asset risk. (This
would be a very interesting advance).

\subsubsection{Market clearing prices}

HARK also has some frameworks for market clearing prices, the Market class.
This has not been integrated with the Portfolio model, but could be in principle.

Other works

The Lucas asset pricing model is a link between macro and finance that could be ported into HARK.
Proposed Integration

In the proposed integration, when the N consumers of the Macro model allocate their savings to “the risky asset”, this interacts with the market model as a demand signal. There are many design questions about how the Macro and Micro/Market models interact. This is one proposal.

\subsubsection{Rebalancing events}

A key concept for the integrated model is rebalancing: when a consumer takes
stock of their situation and adjusts their portfolio allocations accordingly.

Recall that the consumer’s desired portfolio allocation is a function of:
A. Their wealth
B. The expected rate of return and variance of the asset classes
C. Other variables such as permanent income and age that transition
according to equations in the MDP.

Complicating factors in the integrated model include:

That the wealth of a consumer at time t depends on their previous allocation
into the risky asset (measured in dollars), the price at which they bought
the risk asset, and the price at which they would be
able to sell the asset at time t.

Depending on the implementation of the market, the “price” of the asset
may or may not be straightforward, as even the marginal price of the asset
is a function of an endogenously evolving order book. For the purpose of
this writing, we will assume that some form of market price can be derived
from the market model, though the accuracy of this price and the means of
deriving it are a significant area for variation in model design.
The expected rate of return and variance of the risky asset is an emergent
property of the market and not knowable a priori.

In light of the goals of modeling the financial sector in light of bounded
rationality conditions, we will assume for now that: (a) consumers can
compute their wealth at time t based on information from the market, which
includes the price of assets they purchases, and (b) consumers can use
historical information from the market to estimate expected returns and
variance for the risky assets.

In essence, we propose that a rebalance event is when consumers use the
information available to them to determine their portfolio allocations,
adjusting the amount of savings assigned to the risky asset as necessary.
From the market perspective, the rebalance perspective sends a signal,
valued in currency, of how much each consumer wants to buy or sell of the risky asset(s).

Establish market price for purposes of rebalancing

There are two ‘low hanging fruit’ approaches for combining the HARK and ABM
modeling frameworks, both of which are driven by the idea of selective
attention: 1) Intermittent Individual Demand Calculation and Trading
(Intermittent Demand); and 2) Equilibrium Demand Calculation and Intermittent
Delayed Trading (Equilibrium Demand). 

The Intermittent Demand approach assumes that all households have shocks,
which modify their utilities, once a quarter (the Macro Period, which could
be monthly for example), but they  pay attention to the market at random
times (e.g. driven by a hazard rate model) during the subsequent quarter.
When they attend to the market they use the current market price for the
risky asset to calculate their wealth and then based on that ‘current wealth’,
a summary of the recent history of the risky asset (to calculate an expected
return and volatility), and the idiosyncratic shocks that occurred at the
beginning of the quarter, they solve the dynamic programing problem and
determine the amount of their wealth that they want to have in the risky
asset. Comparing this with the amount of wealth they actually have results
in the delta (or number of shares that they want to trade). They then
execute trades using either the emini or HFT ABM model.

The Equilibrium Demand approach assumes, again, that once a quarter individual
level shock occur, which change utilities. Households then come together and
determine an equilibrium price (using the past quarter’s trading history to
calculate a mean and variance), but they do not immediately execute in the
market. There is a residual demand (some individuals hold more of the risky
asset than what they ideally want, at the new equilibrium price, and others
hold less) that is brought to the market at random times during the
subsequent quarter of trading, where households again attend to the market
at different (random) times throughout the quarter.

[John, Mark, Zak] How will consumers establish market price, for the purpose
of the rebalancing calculation described above?

Entering and exiting positions

After each rebalance event, a consumer will attempt to adjust the amount of
their savings invested in the risky asset. 

[John, Mark, Zak] How does the market represent and simulate the entering
and exiting of positions

[Mark conjectures as follows]: 
These are some thoughts on how to integrate the HARK and NetLogo modules

Whereas: The overall simulation will alternate between quarterly equilibrium
simulations in the HARK model and the simulation of intervening daily (or
more frequent) observations of order-book behavior in the NetLogo model
Whereas: Simulation of quarterly equilibrium in the HARK model should
condition on observable history of prior financial-market outcomes, and
Whereas: Simulation of financial-market episodes should condition on
observable history of consumer behavior in the household sector

We can tell the following story:

Each household, in its quarterly rebalancing, generates either a net increase
(buy order flow) or decrease (sell order flow) in its holdings of the risky
asset in the next trading period. The NetLogo model represents the net
effective demand (buy order flow) and effective supply (sell order flow)
through the activities of fundamental buyer and sell agents in the NetLogo
model. The NetLogo model parameterizes these effective demand/supply factors
via NB (the number of fundamental buyers) and NS (the number of fundamental
sellers). (Note to self: I need to check MP’s NetLogo code; these agents
may be called liquidity suppliers and liquidity demanders instead.) For
various reasons, the total new supply/demand might not all arrive in the
financial market abruptly at the start of trading. For example, households
may only learn gradually about macroeconomic conditioning info, due to
limited attention, or they may be liquidity constrained due to other
investments that can only be unwound slowly, etc. In other words,
there are a variety of stories we might tell to explain the gradual
arrival of order flow into the financial market; it matters that story
(or stories) be plausible, but what really matters to the simulation
is that we can throttle the arrival of new order flow, so that it
doesn’t arrive all at once.

In any case, we want a function that translates from the outputs
generated by the HARK model to variables that are implemented in
the NetLogo model. Here is a sketch of how one might do that.
It avoids the programming nuisances of managing the bookkeeping
to keep track of which specific HARK consumer is buying or
selling financial assets, as well as any need for callbacks from the
NetLogo model to the HARK world to manage this bookkeeping. Instead,
the HARK model generates aggregate net buy and sell order flow (for
the financial asset), and the NetLogo model handles how the market
“works” this order flow over the next trading period. 

Suppose the HARK model generates three aggregate values each quarter:
Y1 = aggregate net buy order flow for the risky asset
Y2 = aggregate net sell order flow for the risky asset
Y3 = GDP
(We could handle a longer list of HARK outputs in the translation
functions below, but this is enough to illustrate the proposal.)
Translate these HARK outputs into the NB and NS factors needed by the
NetLogo model as follows (where the Greek superscripts and α values are
parameters to be determined by calibration):

NB = α3Y1βY3γ(P/P*)δeε

and
 
NS = α2Y2βY3γ(P/P*)ζeε 

Note that we allow both NB and NS to depend on GDP (=Y3), although we can
switch off this impact by setting its exponent γ=0. Also, we allow NB and NS
to vary over the financial trading period, by including the ratio of current
market price, P, to fundamental value, P*, as a factor. I think it’s important
for the endogenous dynamics in the financial market that there be feedback
created by the current price. In general, stabilizing speculation should
drive the price down (or up, respectively) when the price gets too high (low),
achieved by choosing ζ>0 to increase the number of sellers on the upside
(and δ=<0 to increase the number of buyers on the downside).
Again, we can switch this effect off by setting δ=0 and ζ=0. I also
include a random term, eε, to prevent the financial dynamics from getting
too deterministic. As with the other terms, we can amplify/attenuate this
effect by our choice of the exponent.

Similarly, we will want the output of the NetLogo simulation to flow back
into the HARK model, so that consumers can condition their choices on
stock market results. There are many statistics we can calculate -- i.e.,
things that consumers might plausibly care about -- but the obvious ones
are risk (realized volatility) and return (natural log of P1/P0). 

[John’s comments]: What is missing is P*. Presumably, one way to get this
is to take the demand schedule for the risky asset (for all of the HARK
households) and find P* by determining the equilibrium price. As a result
the trading would target this as the true value, but there would be market
dynamics which would ultimately result in a new mean and variance at the
end of the quarter of trading, which would then be used by the HARK
households as they determined their new demand functions. This would give
a mechanism for feedback into the HARK model from the market model.

API Design
Inputs for the market model:
Holdings
Wealth
Assets vs. income

Inputs for the consumer:
The current asset price
Remember last asset price because that determines capital gains
Beliefs about:
equity premium, drift parameter --
Variance -- 
Sampled from a distribution?
Taken from market history?
Economist’s challenges

????
It will be difficult to convince the economics reviews if price
is differing very much, instantaneously, from what people believe.


Question: Why is the risky asset valuable?
Consumers care about the total return: dividends plus capital gains
There are other risky assets that don’t pay dividends at all.
The economic model will say:
The target equilibrium price is P_end.
The current price is P_now
The rate of return will be P_end / P_now
Can the economic fundamentals create increasing value to the risky asset?
The ABM may have internal effects on the price
The ABM may be reflecting the real sector
Iterations:
Step #1: Have the ABM settle into something boring
Real economy is perceived to be in a steady state (Aiyagari ‘94)
Step #2: Have a real economy
Krussell-Smith: aggregate shocks.
Research Questions

The integrated model and software tools allow investigation into
several research questions.
Some research questions that have been proposed are described here.

Comparing Cases

First case: Intermittent Demand.

Second case: Equilibrium demand, with steady-state economy. (Aiyagari ‘94)

Third case: Equilibrium demand, with a real economy with aggregate shock.
(Krusell-Smith)



Notes from Sept 4 2020:
Model assumes people know the asset prices at the current moment
If they choose the 
Sequential vs. Simultaneous Rebalancing and Price (dis)Equilibria

How does the market behavior change with different assumptions about the
rebalance schedule of the consumers?

We could run the model with consumers rebalancing on a simultaneous,
regular schedule (quarterly). Or we could run the model with consumers
rebalancing at different rates and at random intervals.
What difference does this make? Do consumers that rebalance
more frequently perform better?



